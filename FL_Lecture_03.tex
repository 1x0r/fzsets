\documentclass{beamer}

\usetheme{Luebeck}
\usecolortheme{crane}

\AtBeginSection[]
{
  \begin{frame}<beamer>
    \frametitle{План раздела \textnumero\thesection}
    \tableofcontents[currentsection]
  \end{frame}
}

\setbeamerfont{block title}{size=\footnotesize}

\usepackage{polyglossia} 		                    % Поддержка многоязычности
\usepackage{fontspec}                             % TrueType-шрифты

\setmainlanguage{russian}
\setotherlanguage{english}
\defaultfontfeatures{Ligatures=TeX, Mapping=tex-text}
\setmainfont[
    Path = fonts/,
    Extension = .ttf,
    BoldFont = PTSerif-Bold,
    ItalicFont = PTSerif-Italic,
    BoldItalicFont = PTSerif-BoldItalic
]{PTSerif-Regular}
\setsansfont[
    Path = fonts/,
    Extension = .ttf,
    BoldFont = PTSans-Bold,
    ItalicFont = PTSans-Italic
]{PTSans-Regular}
\setmonofont[
    Path = fonts/,
    Extension = .ttf
]{PTMono-Regular}

\usepackage{amsthm}
\usepackage{amssymb}
\usepackage{amsthm}
\usepackage{mathtools}
\usepackage{mathcomp}

%\usepackage{nicefrac}
\usepackage[noend]{algorithm2e}
\usepackage{graphicx}
\usepackage{listings}

\lstdefinestyle{mycode}{
  belowcaptionskip=1\baselineskip,
  breaklines=true,
  xleftmargin=\parindent,
  showstringspaces=false,
  basicstyle=\footnotesize\ttfamily,
  keywordstyle=\bfseries,
  commentstyle=\itshape\color{gray!40!black},
  stringstyle=\color{red},
  numbers=left,
  numbersep=5pt,
  numberstyle=\tiny\color{gray},
}

\lstset{escapechar=@,style=mycode}

\title{Лекция \textnumero 3: Принцип обобщения. Нечёткие отношения}
\subtitle{Основы теории нечётких множеств}
\author{Казарян Д.Э.\\{\footnotesize\textcolor{gray}{~ст.~преподаватель}}}
\institute{РУДН, Инженерная академия\\Департамент механики и мехатроники}
\date{9 марта 2017 г.}

\begin{document}
\lstset{language=python}

\frame{\titlepage}

\section{Принцип обобщения}

\begin{frame}\frametitle{Определение принципа обобщения}
    \begin{definition}
        Пусть $X = X_1 \times X_2 \times \dots X_r$, и $\tilde{A}_1, \dots, \tilde{A}_r$ -- нечёткие множества на них, а $f: X \to Y$ (т.\,е. $y = f(x_1,\dots,x_r)$). Согласно \textit{принципу обобщения} возможно задать нечёткое множество $\tilde{B}$ на $Y$:
        \[
            \tilde{B} = \left\{ (y, \mu_{\tilde{B}}(y))\ |\ y = f(x_1,\dots,x_r),\ x_1,\dots,x_r \in X \right\},
        \]
        где
        \[
            \mu_{\tilde{B}}(y) = \begin{cases}
                                    \sup\limits_{x_1,\dots,x_r\in f^{-1}(y)}{\min{\left\{ \mu_{\tilde{{A}_1}}(x_1), \dots, \mu_{\tilde{{A}_r}}(x_r) \right\}}} & \textrm{if}\ f^{-1}(x) \neq \emptyset \\
                                    0 & otherwise
                                 \end{cases}
        \]
    \end{definition}
\end{frame}

\begin{frame}\frametitle{Принцип обобщения при $r=1$}
    При $r=1$
    \[
         \tilde{B} = f(\tilde{A}) =  \left\{ (y, \mu_{\tilde{B}}(y))\ |\ y = f(x),\ x \in X \right\}
    \]
    где
        \[
            \mu_{\tilde{B}}(y) = \begin{cases}
                                    \sup\limits_{x \in f^{-1}(y)}{\mu_{\tilde{{A}}}(x)} & \textrm{if}\ f^{-1}(x) \neq \emptyset \\
                                    0 & otherwise
                                 \end{cases}
        \]
\end{frame}

\begin{frame}\frametitle{Пример принципа обобщения}
    Пусть
    \begin{equation*}
        \begin{align}
            & \tilde{A} = \left\{ (-1, 0.5), (0, 0.8), (1, 1), (2, 0.4) \right\}$ \\
            & f(x) = x^2
        \end{align}
    \end{equation*}
    тогда
    \[
        \tilde{B} = f(\tilde{A}) = \left\{ (0, 0.8), (1, 1), (4, 0.4) \right\}$
    \]
\end{frame}

\subsection{Операции над нечёткими множествами типа 2}

\begin{frame}\frametitle{Нечёткие множества типа 2}
    Заданы 
        \begin{equation*}
        \begin{align}
            \tilde{A}(x) &= \left\{x, \mu_{\tilde{A}}(x) \right\} \\
            \tilde{B}(x) &= \left\{x, \mu_{\tilde{A}}(x) \right\}
        \end{align}
    \end{equation*}
    где
    \begin{equation*}
        \begin{align}
            \mu_{\tilde{A}}(x) &= \left\{ (u_i, \mu_{u_i}(x))\ |\ x \in X,\ u_i, \mu_{u_i}(x) \in \left[0, 1\right] \right\} \\
            \mu_{\tilde{B}}(x) &= \left\{ (v_i, \mu_{v_i}(x))\ |\ x \in X,\ v_i, \mu_{v_i}(x) \in \left[0, 1\right] \right\}
        \end{align}
    \end{equation*}
\end{frame}

\begin{frame}\frametitle{Объединение нечётких множеств типа 2}
    \[
        \mu_{\tilde{A} \cup \tilde{B}}(x) = \left\{ (w, \mu_{\tilde{A} \cup \tilde{B}}(w))\ |\ \max\left\{ u_i, v_j \right\}, u_i, v_j \in \left[0, 1\right] \right\}
    \]
    где
    \[
        \mu_{\tilde{A} \cup \tilde{B}}(w) = \sup\limits_{w=\max{ \left\{u_i, v_i \right\} }}{ \left\{ \mu_{u_i}(x), \mu_{v_i}(x) \right\} }
    \]
\end{frame}

\begin{frame}\frametitle{Пересечение нечётких множеств типа 2}
    \[
        \mu_{\tilde{A} \cap \tilde{B}}(x) = \left\{ (w, \mu_{\tilde{A} \cap \tilde{B}}(w))\ |\ \min\left\{ u_i, v_j \right\}, u_i, v_j \in \left[0, 1\right] \right\}
    \]
    где
    \[
        \mu_{\tilde{A} \cap \tilde{B}}(w) = \sup\limits_{w=\min{ \left\{u_i, v_i \right\} }}{ \left\{ \mu_{u_i}(x), \mu_{v_i}(x) \right\} }
    \]
\end{frame}

\begin{frame}\frametitle{Дополнение для нечёткого множества типа 2}
    \[
         \mu_{\textcent \tilde A}(x) = \left\{ \left[ 1 - u_i, \mu_{\tilde A}(u_i) \right] \right\}
    \]
\end{frame}

\subsection{Алгебраические операции над нечёткими числами}

\begin{frame}\frametitle{Определение нечёткого числа}
    \begin{definition}
        \textit{Нечётким числом} $\tilde{M}$ называется выпуклое нормированное множество $\tilde{M}$ на $\mathbb{R}$ такое, что:
        \begin{enumerate}
            \item $\exists ! x_0: \mu_{\tilde{M}}(x_0) = 1$ ($x_0$ называется средним значением $\tilde{M}$)
            \item $\mu_{\tilde{M}}(x)$ кусочно непрерывна
        \end{enumerate}
    \end{definition}
    Зачастую это определение модифицируют (см., например, числа с трапецеидальной функцией принадлежности)
    \begin{definition}
        Нечёткое числом $\tilde{M}$ является \textit{положительным (отрицательным)}, если $\mu_{\tilde{M}}(x) = 0б\ \forall x < 0\ (\forall x > 0)$ 
    \end{definition}
\end{frame}

\begin{frame}\frametitle{Возрастающие и убывающие бинарные операторы}
    \begin{definition}
        \textit{Возрастающим (убывающим)} бинарным оператором $*$ на $\mathbb{R}$ называется такой оператор, что если $x_1 > y_1$ и $x_2 > y_2$, то $x_1 * x_2 > y_1 * y_2$ ($x_1 * x_2 < y_1 * y_2$)
    \end{definition}
\end{frame}

\begin{frame}\frametitle{Принцип расширения для бинарных операторов}
    \begin{theorem}
        Пусть $\tilde{M}, \tilde{N}$ --- нечёткие числа с непрерывными сюръективными функциями принадлежности $\mu_{\tilde M}, \mu_{\tilde N}: \mathbb{R} \to [0, 1]$, а $*$ непрерывный возрастающий (убывающий) бинарный оператор, тогда $\tilde{M} \circledast \tilde{N}$ --- нечёткое число с непрерывной сюръективной функцией принадлежности $\mu_{\tilde{M} \circledast \tilde{N}}: \mathbb{R} \to [0, 1]$
    \end{theorem}
    
    \begin{theorem}
        Пусть $\tilde{M}, \tilde{N} \in{F(\mathbb{R}})$ --- нечёткие числа с непрерывными функциями принадлежности, тогда применение принципа обобщения $*: \mathbb{R} \circledast \mathbb{R} \to \mathbb{R}$ даёт функцию принадлежности числа $\tilde{M} \circledast \tilde{N}$ вида  $\mu_{\tilde{M} \circledast \tilde{N}} = \sup_{z=x*y}\min{\left\{ \mu_{\tilde{M}}(x), \mu_{\tilde{N}}(y) \right\}}$
    \end{theorem}
    
    \textbf{Замечание} Если оператор $*$ коммутативен (ассоциативен), то и $\circledast$ коммутативен (ассоциативен).
    
\end{frame}

\section{Нечёткие отношения}

\begin{frame}\frametitle{Определение нечёткого отношения}
    \begin{definition}
        Пусть $X, Y \subseteq \mathbb{R}$, тогда
        \[
            \tilde{R} = \left\{(x, y), \mu_{\tilde{R}}(x, y)\ |\ (x,y) \in X \times Y\right\}
        \]
        называется \textit{нечётким отношением} на $X \times Y$
    \end{definition}
    
\end{frame}

\begin{frame}\frametitle{Определение нечёткого отношения (II)}
    \begin{definition}
        Пусть $X, Y \subset \mathbb{R}$ и
        \begin{equation*}
            \begin{align}
                \tilde{A} &= \left\{ (x, \mu_{\tilde{A}}(x))\ |\ x \in X \right\} \\
                \tilde{B} &= \left\{ (y, \mu_{\tilde{B}}(y))\ |\ y \in Y \right\}
            \end{align}
        \end{equation*}
        тогда $\tilde{R}$ называется нечётким отношением на $\tilde{A}, \tilde{B}$, если
        \begin{equation*}
            \left\{
                \begin{align}
                    \mu_{\tilde{R}}(x, y) &\leq \mu_{\tilde{A}}(x),\ \forall (x,y) \in X \times Y \\
                    \mu_{\tilde{R}}(x, y) &\leq \mu_{\tilde{B}}(y),\ \forall (x,y) \in X \times Y
                \end{align}
            \right.
        \end{equation*}
    \end{definition}
\end{frame}

\begin{frame}\frametitle{$\max$-$\min$ композиция}

    \begin{definition}
        Пусть $\tilde{R}_1(x,y)$ и $\tilde{R}_2(x,y)$ --- нечёткие отношения, тогда $\max$-$\min$ композицией называется нечёткое множество
        
        \begin{equation*}
            \begin{align}
               \tilde{R}_1 \circ \tilde{R}_2 = & \left\{ \left[ (x, z), \max_{y}\left\{ \min\left\{ \mu_{\tilde{R}_1}(x, y), 
                                                           \mu_{\tilde{R}_2}(y, z) \right\} \right\} \right] \right. \\
                & \left.\ \ \ |\ x \in X, y\in Y, z \in Z \vphantom{\int}\ \right\}
            \end{align}
        \end{equation*}
        
    \end{definition}
\end{frame}


\end{document}