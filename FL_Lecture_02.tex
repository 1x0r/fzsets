\documentclass{beamer}

\usetheme{Luebeck}
\usecolortheme{crane}

\AtBeginSection[]
{
  \begin{frame}<beamer>
    \frametitle{План раздела \textnumero\thesection}
    \tableofcontents[currentsection]
  \end{frame}
}

\setbeamerfont{block title}{size=\footnotesize}

\usepackage{polyglossia} 		                    % Поддержка многоязычности
\usepackage{fontspec}                             % TrueType-шрифты

\setmainlanguage{russian}
\setotherlanguage{english}
\defaultfontfeatures{Ligatures=TeX, Mapping=tex-text}
\setmainfont[
    Path = fonts/,
    Extension = .ttf,
    BoldFont = PTSerif-Bold,
    ItalicFont = PTSerif-Italic,
    BoldItalicFont = PTSerif-BoldItalic
]{PTSerif-Regular}
\setsansfont[
    Path = fonts/,
    Extension = .ttf,
    BoldFont = PTSans-Bold,
    ItalicFont = PTSans-Italic
]{PTSans-Regular}
\setmonofont[
    Path = fonts/,
    Extension = .ttf
]{PTMono-Regular}

\usepackage{amsthm}
\usepackage{amssymb}
\usepackage{amsthm}
\usepackage{mathtools}
\usepackage{mathcomp}

%\usepackage{nicefrac}
\usepackage[noend]{algorithm2e}
\usepackage{graphicx}
\usepackage{listings}

\lstdefinestyle{mycode}{
  belowcaptionskip=1\baselineskip,
  breaklines=true,
  xleftmargin=\parindent,
  showstringspaces=false,
  basicstyle=\footnotesize\ttfamily,
  keywordstyle=\bfseries,
  commentstyle=\itshape\color{gray!40!black},
  stringstyle=\color{red},
  numbers=left,
  numbersep=5pt,
  numberstyle=\tiny\color{gray},
}

\lstset{escapechar=@,style=mycode}

\title{Лекция \textnumero 2: Типы нечётких множеств. Нормы}
\subtitle{Основы теории нечётких множеств}
\author{Казарян Д.Э.\\{\footnotesize\textcolor{gray}{~ст.~преподаватель}}}
\institute{РУДН, Инженерная академия\\Департамент механики и мехатроники}
\date{2 марта 2017 г.}

\begin{document}
\lstset{language=python}

\frame{\titlepage}

\section{Типы нечётких множеств}

\begin{frame}\frametitle{Нечёткое множество типа $m$}

    Рассмотренные выше нечёткие множества c чёткими функциями принадлежности называются нечёткими множествами \textit{типа 1}.
    
    \begin{definition}
        Нечётким множеством \textit{типа $m$} называется нечёткое множество на множестве $X$, значения функции принадлежности которого являются нечётким множеством типа $m-1$, $m > 1$ на интервале $[0, 1]$.
    \end{definition}
    
\end{frame}

\begin{frame}\frametitle{Вероятностное нечёткое множество}

    \begin{definition}
        \textit{Вероятностным} нечётким множеством называется нечёткое множество $A$ на множестве $X$ с функцией принадлежности
        \[
            \mu_A : X \times \Omega \ni (x, \omega) \to \mu_A(x, \omega) \in \Omega_C
        \]
    \end{definition}
    
\end{frame}

\begin{frame}\frametitle{Интуиционистское нечёткое множество}

    \begin{definition}
        \textit{Интуиционистским} нечётким множеством называется нечёткое множество $A$ на множестве $X$, состоящее из трёхэлементных кортежей
        \[
            A = \left\{ \left(x, \mu_A(x), v_A(x)\right)\ |\ x \in X \right\}
        \]
        где $v_A(x)$ -- функция, показывающая степень непринадлежности, причём
        \[
            0 \leq \mu_A(x) + v_A(x) \leq 1
        \]
    \end{definition}
    
\end{frame}

\section{Операции над нечёткими множествами}

\subsection{$t$-норма и $t$-конорма ($s$-норма)}

\begin{frame}\frametitle{$t$-норма}
    
    \begin{definition}
        \textit{$t$-нормой} называется коммутативная, ассоциативная, монотонная двузначная функция $[0, 1] \times [0, 1] \to [0, 1]$, удовлетворяющая условиям:
        \begin{enumerate}
            \item $t(0,0) = 0$; $t(\mu_{\tilde{A}}(x), 1) = t(1, \mu_{\tilde{A}}(x)) = \mu_{\tilde{A}}(x),\ x \in X$
            \item $t(\mu_{\tilde{A}}(x), \mu_{\tilde{B}}(x)) \leq t(\mu_{\tilde{C}}(x), \mu_{\tilde{D}}(x))$, если $t(\mu_{\tilde{A}}(x)) \leq t(\mu_{\tilde{C}}(x))$ и $t(\mu_{\tilde{B}}(x)) \leq t(\mu_{\tilde{D}}(x))$
            \item $t(\mu_{\tilde{A}}(x), \mu_{\tilde{B}}(x)) = t(\mu_{\tilde{B}}(x), \mu_{\tilde{A}}(x))$
            \item $t(\mu_{\tilde{A}}(x), t(\mu_{\tilde{B}}(x), \mu_{\tilde{C}}(x))) = t(t(\mu_{\tilde{A}}(x), \mu_{\tilde{B}}(x)), \mu_{\tilde{C}}(x)))$
        \end{enumerate}
    \end{definition}
    
    $t$-норма описывает класс функций пересечения для нечётких множеств.
\end{frame}

\begin{frame}\frametitle{$t$-конорма ($s$-норма)}
    
    \begin{definition}
        \textit{$t$-конормой ($s$-нормой)} называется коммутативная, ассоциативная, монотонная двузначная функция $[0, 1] \times [0, 1] \to [0, 1]$, удовлетворяющая условиям:
        \begin{enumerate}
            \item $s(1,1) = 1$; $s(\mu_{\tilde{A}}(x), 0) = s(0, \mu_{\tilde{A}}(x)) = \mu_{\tilde{A}}(x),\ x \in X$
            \item $s(\mu_{\tilde{A}}(x), \mu_{\tilde{B}}(x)) \leq s(\mu_{\tilde{C}}(x), \mu_{\tilde{D}}(x))$, если $s(\mu_{\tilde{A}}(x)) \leq s(\mu_{\tilde{C}}(x))$ и $s(\mu_{\tilde{B}}(x)) \leq s(\mu_{\tilde{D}}(x))$
            \item $s(\mu_{\tilde{A}}(x), \mu_{\tilde{B}}(x)) = s(\mu_{\tilde{B}}(x), \mu_{\tilde{A}}(x))$
            \item $s(\mu_{\tilde{A}}(x), s(\mu_{\tilde{B}}(x), \mu_{\tilde{C}}(x))) = s(s(\mu_{\tilde{A}}(x), \mu_{\tilde{B}}(x)), \mu_{\tilde{C}}(x)))$
        \end{enumerate}
    \end{definition}
    
    $t$-конорма описывает класс функций объединения для нечётких множеств.
\end{frame}

\begin{frame}\frametitle{Связь $t$- и $s$-нормы}
    
    \begin{itemize}
        \item $t$- и $s$-норма логически дуальны:
        \[
            t(\mu_{\tilde{A}}(x), \mu_{\tilde{B}}(x)) = 1 - s(1 - \mu_{\tilde{A}}(x), 1 - \mu_{\tilde{B}}(x)))
        \]
        \item Обобщение законов де Моргана, если определить оператор дополнения как $n(\mu_{\tilde{A}}) = 1 - \mu_{\tilde{A}}$
        \[
        \begin{align}
            s(\mu_{\tilde{A}}(x), \mu_{\tilde{B}}(x)) &= n(t(n(\mu_{\tilde{A}}(x)), n(\mu_{\tilde{B}}(x)))) \mathrm{and} \\
            t(\mu_{\tilde{A}}(x), \mu_{\tilde{B}}(x)) &= n(s(n(\mu_{\tilde{A}}(x)), n(\mu_{\tilde{B}}(x)))), x \in X
        \end{align}
        \]
    \end{itemize}

\end{frame}

\subsection{Непараметрические $t$- и $s$-нормы}

\begin{frame}\frametitle{$t_w$-, $s_w$-нормы}
    \begin{itemize}
        \item Жёсткое произведение (drastic product):
        \[ \hspace*{-1cm} 
            t_w(\mu_{\tilde{A}}(x), \mu_{\tilde{B}}(x)) = 
                \begin{cases}
                    \min{\left\{ \mu_{\tilde{A}}(x), \mu_{\tilde{B}}(x) \right\}} & \mathrm{if} \max{\left\{ \mu_{\tilde{A}}(x), \mu_{\tilde{B}}(x) \right\}} = 1 \\
                    0 & otherwise
                \end{cases}
        \]
        \item  Жёсткая сумма (drastic sum):
        \[ \hspace*{-1cm} 
            s_w(\mu_{\tilde{A}}(x), \mu_{\tilde{B}}(x)) = 
                \begin{cases}
                    \max{\left\{ \mu_{\tilde{A}}(x), \mu_{\tilde{B}}(x) \right\}} & \mathrm{if} \min{\left\{ \mu_{\tilde{A}}(x), \mu_{\tilde{B}}(x) \right\}} = 0 \\
                    1 & otherwise
                \end{cases}
        \]
    \end{itemize}
\end{frame}

\begin{frame}\frametitle{$t_1$-, $s_1$-нормы}
    \begin{itemize}
        \item Ограниченная разность (bounded difference):
        \[
            t_1(\mu_{\tilde{A}}(x), \mu_{\tilde{B}}(x)) = 
                \max{\left\{0, \mu_{\tilde{A}}(x) + \mu_{\tilde{B}}(x) - 1 \right\} }
        \]
        \item  Ограниченная сумма (bounded sum):
        \[
            s_1(\mu_{\tilde{A}}(x), \mu_{\tilde{B}}(x)) = 
                \min{\left\{1, \mu_{\tilde{A}}(x) + \mu_{\tilde{B}}(x) \right\} }
        \]
    \end{itemize}
\end{frame}


\begin{frame}\frametitle{$t_{1.5}$-, $s_{1.5}$-нормы}
    \begin{itemize}
        \item Произведение Эйнштейна (Einstein product):
        \[
            t_{1.5}(\mu_{\tilde{A}}(x), \mu_{\tilde{B}}(x)) = 
                \frac{\mu_{\tilde{A}}(x) \cdot \mu_{\tilde{B}}(x)}{2 - (\mu_{\tilde{A}}(x) + \mu_{\tilde{B}}(x) - \mu_{\tilde{A}}(x) \cdot \mu_{\tilde{B}}(x))}
        \]
        \item Сумма Эйнштейна (Einstein sum):
        \[
            s_{1.5}(\mu_{\tilde{A}}(x), \mu_{\tilde{B}}(x)) = \frac{\mu_{\tilde{A}}(x) + \mu_{\tilde{B}}(x)}{1 + \mu_{\tilde{A}}(x) \cdot \mu_{\tilde{B}}(x)}
        \]
    \end{itemize}
\end{frame}

\begin{frame}\frametitle{$t_{2}$-, $s_{2}$-нормы}
    \begin{itemize}
        \item Алгебраическое произведение (algebraic product):
        \[
            t_{2}(\mu_{\tilde{A}}(x), \mu_{\tilde{B}}(x)) = \mu_{\tilde{A}}(x) \cdot \mu_{\tilde{B}}(x)
        \]
        \item Алгебраическая сумма (algebraic sum):
        \[
            s_{2}(\mu_{\tilde{A}}(x), \mu_{\tilde{B}}(x)) = \mu_{\tilde{A}}(x) + \mu_{\tilde{B}}(x) - \mu_{\tilde{A}}(x) \cdot \mu_{\tilde{B}}(x)
        \]
    \end{itemize}
\end{frame}


\begin{frame}\frametitle{$t_{2.5}$-, $s_{2.5}$-нормы}
    \begin{itemize}
        \item Произведение Хамахера (Hamacher product):
        \[
            t_{2.5}(\mu_{\tilde{A}}(x), \mu_{\tilde{B}}(x)) = 
                \frac{\mu_{\tilde{A}}(x) \cdot \mu_{\tilde{B}}(x)}{(\mu_{\tilde{A}}(x) + \mu_{\tilde{B}}(x) - \mu_{\tilde{A}}(x) \cdot \mu_{\tilde{B}}(x))}
        \]
        \item Сумма Хамахера (Hamacher sum):
        \[
            s_{2.5}(\mu_{\tilde{A}}(x), \mu_{\tilde{B}}(x)) = \frac{\mu_{\tilde{A}}(x) + \mu_{\tilde{B}}(x) - 2 \mu_{\tilde{A}}(x) \cdot \mu_{\tilde{B}}(x)}{1 - \mu_{\tilde{A}}(x) \cdot \mu_{\tilde{B}}(x)}
        \]
    \end{itemize}
\end{frame}

\begin{frame}\frametitle{$t_{3}$-, $s_{3}$-нормы}
    \begin{itemize}
        \item Минимум (minimum):
        \[
            t_{3}(\mu_{\tilde{A}}(x), \mu_{\tilde{B}}(x)) = \min{\left\{ \mu_{\tilde{A}}(x), \mu_{\tilde{B}}(x) \right\}}
        \]
        \item Максимум (maximum):
        \[
            s_{3}(\mu_{\tilde{A}}(x), \mu_{\tilde{B}}(x)) = \max{\left\{ \mu_{\tilde{A}}(x), \mu_{\tilde{B}}(x) \right\}}
        \]
    \end{itemize}
\end{frame}

\begin{frame}\frametitle{Отношение между непараметрическими нормами}
    \begin{itemize}
        \item Отношение между $t$-нормами:
        \[
            t_{w} \leq t_1 \leq t_{1.5} \leq t_2 \leq t_{2.5} \leq t_3
        \]
        \item Отношение между $s$-нормами:
        \[
            s_{3} \leq s_{2.5} \leq s_{2} \leq s_{1.5} \leq s_{1} \leq s_w
        \]
    \end{itemize}
\end{frame}

\subsection{Параметрические $t$- и $s$-нормы}

\begin{frame}\frametitle{Пересечение и объединение по Хамахеру}

    \begin{itemize}
        \item Пересечение по Хамахеру:
        \[
            \mu_{\tilde{A} \cap \tilde{B}}(x) = \frac{\mu_{\tilde{A}}(x) \mu_{\tilde{B}}(x)}{\gamma + (1-\gamma)(\mu_{\tilde{A}}(x) + \mu_{\tilde{B}}(x) - \mu_{\tilde{A}}(x) \cdot \mu_{\tilde{B}}(x))},\ \gamma \geq 0
        \]
        \item Объединение по Хамахеру:
        \[
            \mu_{\tilde{A} \cup \tilde{B}}(x) = \frac{(1-\gamma')\mu_{\tilde{A}}(x) \mu_{\tilde{B}}(x) + \mu_{\tilde{A}}(x) + \mu_{\tilde{B}}(x)}{1 + \gamma' \mu_{\tilde{A}}(x) \cdot \mu_{\tilde{B}}(x)},\ \gamma' \geq -1
        \]
    \end{itemize}
    
\end{frame}

\begin{frame}\frametitle{Пересечение и объединение по Ягеру}

    \begin{itemize}
        \item Пересечение по Ягеру:
        \[ \hspace*{-1cm} 
            \mu_{\tilde{A} \cap \tilde{B}}(x) = 1 - \min{ \left\{ 1, \left( (1 - \mu_{\tilde{A}}(x))^p + (1 - \mu_{\tilde{B}}(x)))^p \right)^{1/p} \right\}}, p \geq 1
        \]
        \item Объединение по Ягеру:
        \[
            \mu_{\tilde{A} \cap \tilde{B}}(x) = \min{ \left\{ 1, \left( \mu_{\tilde{A}}(x)^p + \mu_{\tilde{B}}(x))^p \right)^{1/p} \right\}}, p \geq 1
        \]
    \end{itemize}
    
\end{frame}

\end{document}