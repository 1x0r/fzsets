\documentclass{beamer}

\usetheme{Luebeck}
\usecolortheme{crane}

\AtBeginSection[]
{
  \begin{frame}<beamer>
    \frametitle{План раздела \textnumero\thesection}
    \tableofcontents[currentsection]
  \end{frame}
}

\setbeamerfont{block title}{size=\footnotesize}

\usepackage{polyglossia} 		                    % Поддержка многоязычности
\usepackage{fontspec}                             % TrueType-шрифты

\setmainlanguage{russian}
\setotherlanguage{english}
\defaultfontfeatures{Ligatures=TeX, Mapping=tex-text}
\setmainfont[
    Path = fonts/,
    Extension = .ttf,
    BoldFont = PTSerif-Bold,
    ItalicFont = PTSerif-Italic,
    BoldItalicFont = PTSerif-BoldItalic
]{PTSerif-Regular}
\setsansfont[
    Path = fonts/,
    Extension = .ttf,
    BoldFont = PTSans-Bold,
    ItalicFont = PTSans-Italic
]{PTSans-Regular}
\setmonofont[
    Path = fonts/,
    Extension = .ttf
]{PTMono-Regular}

\usepackage{amsthm}
\usepackage{amssymb}
\usepackage{amsthm}
\usepackage{mathtools}
\usepackage{mathcomp}

%\usepackage{nicefrac}
\usepackage[noend]{algorithm2e}
\usepackage{graphicx}
\usepackage{listings}

\lstdefinestyle{mycode}{
  belowcaptionskip=1\baselineskip,
  breaklines=true,
  xleftmargin=\parindent,
  showstringspaces=false,
  basicstyle=\footnotesize\ttfamily,
  keywordstyle=\bfseries,
  commentstyle=\itshape\color{gray!40!black},
  stringstyle=\color{red},
  numbers=left,
  numbersep=5pt,
  numberstyle=\tiny\color{gray},
}

\lstset{escapechar=@,style=mycode}

\title{Лекция \textnumero 4: Нечёткие графы. Нечёткий математический анализ}
\subtitle{Основы теории нечётких множеств}
\author{Казарян Д.Э.\\{\footnotesize\textcolor{gray}{~ст.~преподаватель}}}
\institute{РУДН, Инженерная академия\\Департамент механики и мехатроники}
\date{23 марта 2017 г.}

\begin{document}
\lstset{language=python}

\frame{\titlepage}

\section{Нечёткие графы}

\begin{frame}\frametitle{Нечёткий граф}
    \begin{definition}
        Пусть $E$ --- чёткое множество вершин. \textit{Нечёткий граф} определяется как
        \[
            \tilde{G}(x_i, x_j) = \left\{ \left( \left( x_i. x_j \right), \mu_{\tilde{G}}\left( x_i. x_j \right) \right)\ |\ \left( x_i. x_j \right) \in E \times E \right\}
        \]
    \end{definition}
\end{frame}


\begin{frame}\frametitle{Нечёткий подграф}
    \begin{definition}
        Пусть $\tilde{H}\left( x_i. x_j \right)$ --- \textit{нечёткий подграф} графа $\tilde{G}\left( x_i. x_j \right)$, если
        \[
            \mu_{\tilde{H}}\left( x_i. x_j \right) \leq \mu_{\tilde{G}}\left( x_i. x_j \right), \forall \left( x_i. x_j \right) \in E \times E
        \]
    \end{definition}
\end{frame}

\begin{frame}\frametitle{Путь в графе}
    \begin{definition}
        \textit{Путь} в нечётком графе $\tilde{G}(x_i, x_j)$ --- это последовательность различных вершин $x_0, x_1, \dots, x_n,\ \forall (x_i, x_j),\ \mu_{\tilde{G}} > 0$.
    \end{definition}

    \begin{itemize}
        \item \textit{Сила пути} в нечётком графе $\tilde{G}(x_i, x_j)$ --- это $\min\left\{ \mu_{\tilde{G}}\left(x_i, x_{i+1}\right) \right\},\ i = \overline{0, i-1}$
        \item \textit{Длина пути} -- число вершин, входящих в путь.
        \item \textit{Цикл} -- путь, в котором $x_0 = x_n,\ n \geq 2$
    \end{itemize}
    
\end{frame}

\begin{frame}\frametitle{$\mu$-длина и $\mu$-расстояние}
    \begin{definition}
        \textit{$\mu$-длина} пути в нечётком графе --- это
        \[
            L(p) = \sum_{i=0}^{n-1}\frac{1}{\mu(x_i, x_{i+1})}
        \]
    \end{definition}
    
    \begin{definition}
        \textit{$\mu$-расстояние} $d(x_i, x_j)$ между двумя узлами $x_i, x_j$ --- это наименьшая $\mu$-длина пути среди всех путей между $x_i, x_j$:
        \[
            d(x_i, x_j) = \min_p L(p),\ \forall p = x_i, \dots, x_j
        \]
    \end{definition}
\end{frame}

\begin{frame}\frametitle{Лес}
    \begin{itemize}
        \item Два узла называются \textit{связанными}, если между ними существует путь
        \item Нечёткий граф называется \textit{лесом}, если в нём нет циклов
        \item Связный лес называется \textit{деревом}
    \end{itemize}

\end{frame}

\section{Нечёткий математический анализ}

\begin{frame}\frametitle{Нечёткая функция}
    \begin{itemize}
        \item Нечёткая функция --- это обобщение чёткой функции
        \item Чёткая функция: $f: S \to D$
        \item Нечёткая функция имеет степени нечёткости:
        \begin{enumerate}
            \item чёткое отображение нечёткого множества формирует нечёткое множество
            \item нечёткое отображение чёткого множества (такие функции называются фаззифицирующими)
            \item чёткая функция может обладать нечёткими свойствами
        \end{enumerate}
    \end{itemize}
\end{frame}

\begin{frame}\frametitle{Определения нечёткой функции}
    \begin{definition}
        \textit{Классическая функция} $f: X \to Y$ отображает нечёткую область определения $\tilde{A}$ на $X$ в нечёткое множество значений $\tilde{B}$ на $Y$ тогда и только тогда, когда
        \[
            \forall x \in X, \mu_{\tilde{B}}(f(x)) \geq \mu_{\tilde{A}}(x)
        \]
    \end{definition}
    
    \begin{definition}
        Пусть даны пространства $X$ и $Y$, $\tilde{P}(Y)$ --- множество всех нечётких множеств на $Y$. Тогда $\tilde{f}(x): X \to \tilde{P}(Y)$ --- нечёткая функция, тогда и только тогда, когда
        \[
            \forall (x, y) \in X \times Y, \mu_{\tilde{f}(x)}(y) = \mu_{\tilde{P}}(x, y)
        \]
    \end{definition}
\end{frame}

\begin{frame}\frametitle{Максимизирующее множество}
    \begin{definition}
        Пусть $f: X \to \mathbb{R}$ ограничена снизу и сверху. Тогда нечёткое множество $\tilde{M} = \{ (x, \mu_{\tilde{M}}(x))\ |\ x\inX \}$,
        \[
            \mu_{\tilde{M}}(x) = \frac{f(x) - \mathrm{inf}(f)}{\mathrm{sup}(x) - \mathrm{inf}(f)}
        \]
        называется максимизирующим множеством
    \end{definition}
    
\end{frame}
\begin{frame}\frametitle{Нечёткий максимум}
    \begin{definition}
        Пусть $\tilde{f}: X \to \mathbb{R}$ определена на чётком множестве $D$. Тогда \textit{нечётким максимумом} $\tilde{f}(x)$ называется
        \[
            \tilde{M} = \max_{x \in X} \tilde{f}(x) = \{ \sup{\tilde{f}(x), \mu_{\tilde{M}}(x))\ |\ x \in D} \}
        \]
    \end{definition}    
\end{frame}

\end{document}